\documentclass[9pt, compress]{beamer}\usepackage[]{graphicx}\usepackage[]{color}
%% maxwidth is the original width if it is less than linewidth
%% otherwise use linewidth (to make sure the graphics do not exceed the margin)
\makeatletter
\def\maxwidth{ %
  \ifdim\Gin@nat@width>\linewidth
    \linewidth
  \else
    \Gin@nat@width
  \fi
}
\makeatother

\definecolor{fgcolor}{rgb}{0.345, 0.345, 0.345}
\newcommand{\hlnum}[1]{\textcolor[rgb]{0.686,0.059,0.569}{#1}}%
\newcommand{\hlstr}[1]{\textcolor[rgb]{0.192,0.494,0.8}{#1}}%
\newcommand{\hlcom}[1]{\textcolor[rgb]{0.678,0.584,0.686}{\textit{#1}}}%
\newcommand{\hlopt}[1]{\textcolor[rgb]{0,0,0}{#1}}%
\newcommand{\hlstd}[1]{\textcolor[rgb]{0.345,0.345,0.345}{#1}}%
\newcommand{\hlkwa}[1]{\textcolor[rgb]{0.161,0.373,0.58}{\textbf{#1}}}%
\newcommand{\hlkwb}[1]{\textcolor[rgb]{0.69,0.353,0.396}{#1}}%
\newcommand{\hlkwc}[1]{\textcolor[rgb]{0.333,0.667,0.333}{#1}}%
\newcommand{\hlkwd}[1]{\textcolor[rgb]{0.737,0.353,0.396}{\textbf{#1}}}%

\usepackage{framed}
\makeatletter
\newenvironment{kframe}{%
 \def\at@end@of@kframe{}%
 \ifinner\ifhmode%
  \def\at@end@of@kframe{\end{minipage}}%
  \begin{minipage}{\columnwidth}%
 \fi\fi%
 \def\FrameCommand##1{\hskip\@totalleftmargin \hskip-\fboxsep
 \colorbox{shadecolor}{##1}\hskip-\fboxsep
     % There is no \\@totalrightmargin, so:
     \hskip-\linewidth \hskip-\@totalleftmargin \hskip\columnwidth}%
 \MakeFramed {\advance\hsize-\width
   \@totalleftmargin\z@ \linewidth\hsize
   \@setminipage}}%
 {\par\unskip\endMakeFramed%
 \at@end@of@kframe}
\makeatother

\definecolor{shadecolor}{rgb}{.97, .97, .97}
\definecolor{messagecolor}{rgb}{0, 0, 0}
\definecolor{warningcolor}{rgb}{1, 0, 1}
\definecolor{errorcolor}{rgb}{1, 0, 0}
\newenvironment{knitrout}{}{} % an empty environment to be redefined in TeX

\usepackage{alltt}
\usepackage[utf8]{inputenc}
\usepackage{color}


%%% ----- Theme ----------------------------------------------------------------
\usetheme{default}
\useoutertheme[subsection=false]{miniframes}



% move navigation to footer
\setbeamertemplate{headline}{}
\makeatletter
\setbeamertemplate{footline}
  {%
  \begin{beamercolorbox}{section in head/foot}
    \vskip2pt\insertnavigation{\paperwidth}\vskip5pt
  \end{beamercolorbox}%
  }
\makeatother

% no navigation bar
\beamertemplatenavigationsymbolsempty

% some colors
\definecolor{foreground}{RGB}{0, 0, 0}
\definecolor{background}{RGB}{255,255,255}
\definecolor{title}{RGB}{77, 123, 150}
\definecolor{gray}{RGB}{116,116,116}
\definecolor{hilight}{RGB}{228, 97, 0}
\setbeamercolor{section in head/foot}{fg = title}
% set colors
\setbeamercolor{titlelike}{fg=title}
\setbeamercolor{subtitle}{fg=title}
\setbeamercolor{institute}{fg=gray}
\setbeamercolor{normal text}{fg=foreground,bg=background}
% set colors for itemize
\setbeamercolor{item}{fg=foreground} % color of bullets
\setbeamercolor{subitem}{fg=gray}
\setbeamercolor{itemize/enumerate subbody}{fg=gray}


%%% ----- Metadata -------------------------------------------------------------
\title{A (brief) introduction to ordination and the vegan package}
\date{SEFS9, July 5th 2015}
\author{Eduard Szöcs}
\institute{Institute for Environmental Sciences - University of Koblenz-Landau \\[1em] 
  \includegraphics[width=2.5cm]{fig/Institut.png}}



%%% ----- Macros ---------------------------------------------------------------
%%% Footnotetext without numbering
\makeatletter
\def\blfootnote{\gdef\@thefnmark{}\@footnotetext}
\makeatother






%%% ----------------------------------------------------------------------------
\IfFileExists{upquote.sty}{\usepackage{upquote}}{}
\begin{document}



\maketitle

\begin{frame}
\frametitle{}
\end{frame}


%%% ----------------------------------------------------------------------------
\section{Datasets}
\begin{frame}[plain]{}
\begin{center}
  \Huge \color{title} Datasets
\end{center}
\end{frame}

\subsection{}
\begin{frame}
\frametitle{Demonstration: Salinization and Pestices}

\blfootnote{\color{gray} Szöcs, E., Kefford, B.J., Schäfer, R.B., 2012. Is there an interaction of the effects of salinity and pesticides on the community structure of macroinvertebrates? Science of the Total Environment 437, 121–126.}
\end{frame}


\begin{frame}
\frametitle{Demonstration: Salinization and Pestices}
\begin{knitrout}\footnotesize
\definecolor{shadecolor}{rgb}{0.969, 0.969, 0.969}\color{fgcolor}\begin{kframe}
\begin{alltt}
\hlkwd{setwd}\hlstd{(}\hlstr{'your/workingdirectory'}\hlstd{)}
\end{alltt}
\end{kframe}
\end{knitrout}
% setwd('/home/edisz/Documents/Uni/Projects/PHD/CONFERENCES/SEFS9_Geneva/3-IntroVeganPackage/data/')
\begin{knitrout}\footnotesize
\definecolor{shadecolor}{rgb}{0.969, 0.969, 0.969}\color{fgcolor}\begin{kframe}
\begin{alltt}
\hlstd{abu} \hlkwb{<-} \hlkwd{read.table}\hlstd{(}\hlstr{'melbourneAbu.csv'}\hlstd{,} \hlkwc{sep} \hlstd{=} \hlstr{';'}\hlstd{)}

\hlstd{env} \hlkwb{<-} \hlkwd{read.table}\hlstd{(}\hlstr{'melbourneEnv.csv'}\hlstd{,} \hlkwc{sep} \hlstd{=} \hlstr{';'}\hlstd{)}
\end{alltt}
\end{kframe}
\end{knitrout}


\end{frame}


\subsection{}
\begin{frame}
\frametitle{Exercise: Doubs river fish communities}
\end{frame}



%%% ----------------------------------------------------------------------------
\section{Constrained Ordination}
\begin{frame}[plain]{}
\begin{center}
  \Huge \color{title} Unconstrained Ordination
\end{center}
\end{frame}


\subsection{}
\begin{frame}
\frametitle{Unconstrained Ordination}
\begin{itemize}
  \item Principal Components Analysis (PCA)
  \item Nonmetric Multidimensional Scaling (NMDS)
\end{itemize}
\end{frame}



\bgroup
\setbeamercolor{background canvas}{bg=title}
\begin{frame}[plain]{}
\begin{center}
  \color{background}
  \Huge \textbf{Your turn}! \\[1em]
  \large
  Load the doubs data. \\
  Is there a downstream gradient?
\end{center}
\end{frame}
\egroup


%%% ----------------------------------------------------------------------------
\section{Constrained Ordination}
\begin{frame}[plain]{}
\begin{center}
  \Huge \color{title} Constrained Ordination
\end{center}
\end{frame}


\subsection{}
\begin{frame}
\frametitle{Constrained Ordination}
\begin{itemize}
  \item Redundany analysis (RDA)
  \item Transformation-based RDA
  \item Distance-based RDA
  \item Permutation Tests
\end{itemize}
\end{frame}

\bgroup
\setbeamercolor{background canvas}{bg=title}
\begin{frame}[plain]{}
\begin{center}
  \color{background}
  \Huge \textbf{Your turn}! \\[1em]
  \large
  Using the doubs data. \\
\end{center}
\end{frame}
\egroup


%%% ----------------------------------------------------------------------------
\section{Permutation Tests}
\begin{frame}[plain]{}
\begin{center}
  \Huge \color{title} Permutation Tests
\end{center}
\end{frame}


\bgroup
\setbeamercolor{background canvas}{bg=title}
\begin{frame}[plain]{}
\begin{center}
  \color{background}
  \Huge \textbf{Your turn}! \\[1em]
  \large
  Using the doubs data. \\
\end{center}
\end{frame}
\egroup

%%% ----------------------------------------------------------------------------
\end{document}
